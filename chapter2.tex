\chapter{LITERATURE REVIEW}

This chapter will cover the articles that we used to help design and develop the system and the structure of the project. Some of the points that are made in these articles will be used to help better the process of our project by improving from the mistakes discovered by them.

\section{Gamification for education purposes: What are the factors contributing to varied effectiveness? \cite{paper_1}}
This paper explores the effectiveness of using gamification for educational purpose by doing content analysis on 44 articles that the author have selected.

According to the paper, previous researchers deemed engagement as the main factor on how to measure the effectiveness of gamification but the topic still remains controversial as the results from previous empirical studies as varied.

Results from the content analysis showed that the result from gamified learning are also varied depending on the user. Some users showed improvements in understanding of the subject while others are less motivated and performed worse than control group.

The future work is to experiment the effectiveness of gamification on different types of users.

\newpage
\section{Practices, purposes and challenges in integrating gamification using technology: A mixed-methods study on university academics \cite{paper_2}}
This paper explores how gamification are utilized in higher education and its challenges.

Five main themes were found to be critical in enhancing the process; motivating the students, engaging the students, facilitating problem-solving skill, facilitating interactions, and achieving specific goals.

Authors pointed out a problem which is that academics are not able to utilize gamification fully is due to the lack of understanding the process which leads to poor execution.

The future work is to implement the themes to observe whether those themes are critical to the gamification process.