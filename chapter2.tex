\chapter{LITERATURE REVIEW}

In this chapter, the literature review will be divided into 3 parts: difficulty in learning programming language, game designs and adaptive learning algorithms. We will review the articles and present the advantages, disadvantages, and the summary of the paper. Some points made in the articles will be used to support and improve our project.

\section{Difficulty in Learning Programming Language}
\subsection{Factors Contributing to the Difficulties in Teaching and Learning of Computer Programming: A Literature Review \cite{Factors}}
This paper’s main objective is to point out possible reasons why the dropout rate in programming courses are high, even in the introductory ones. Previous researches that were cited by the author stated that the reasons are that the learners lack problem-solving skills. The author sought to confirm the reasons by reviewing research papers and singling out possible reasons.

According to the author, the students have problems understanding the phases of programming such as designing the program and overall have a lack of understanding the fundamental skills. 

Other possible reasons that contribute are the static learning materials and learning methods. Students are shown to be more receptive to visual representations of the code inner workings and preferred working with examples more than lecture-only style. There were also some intangible factors such as personal traits and attitudes as well.

Our project will be taking these factors into account while developing the application to ensure that the learners will benefit from using the application.
\subsection{What about a simple language? Analyzing the difficulties in learning to program \cite{Analyze}}
The objective of the paper is to differentiate the students’ abilities in writing programs between two languages, Java and Python, and observe whether the difficulty in programming languages will impact the difficulty in learning how to program.

The result from the experiment showed that students that learned programming via Python made less syntax and logic errors, which resulted with more programs written in Python were able to run and satisfy the requirements. 

The authors also ran a follow-up experiment to see whether the Python users will be able to migrate to Java properly, the result showed that students with Python background have an evident advantage on other students who did not have it. Students also reported that Python gave an easier to understand concept and basics for programming.

We decided to take this into account and chose C language since it is the middle ground between Python and Java in terms of difficulty.\newline

\noindent\hspace{1.5em}
From the papers above, we can concluded that the difficulty of learning a programming language lies in the lack of skills from the students, the undesirable learning methods such as pure lectures, and the rules of the language itself.

\noindent\hspace{1.5em}
While it is impossible to lower the rule's difficulty, it is possible to help students learn programming better by teaching them to skills needed to understand how to program and making the resources engaging to learn from.

\section{Game Designs}
\subsection{Gamification for education purposes: What are the factors contributing to varied effectiveness? \cite{EffectFactor}}
This paper explores the effectiveness of using gamification for educational purpose by doing content analysis on 44 articles that the author have selected.

According to the paper, previous researchers deemed engagement as the main factor on how to measure the effectiveness of gamification but the topic still remains controversial as the results from previous empirical studies as varied.

Results from the content analysis showed that the result from gamified learning are also varied depending on the user. Some users showed improvements in understanding of the subject while others are less motivated and performed worse than control group.

The future work is to experiment the effectiveness of gamification on different types of users.

\subsection{Designing Engaging Games for Education: A Systematic Literature Review on Game Motivators and Design Principles \cite{Design}}
The authors stated that for effective helping, sufficient learner engagement is needed. The author intended to reinforce the previous researchers' statement, that various game designs and motivators have proved to be effective, by doing systematic literature reviews.

After the reviews, the authors proposed interlinked taxonomies: game motivators and educational game design principles. The authors found that the founded game motivators and design principles do indicate strong support of previous researches.

We will be utilizing the founded design principles for the application.

\subsection{The Effects of Game Design on Learning Outcomes \cite{Effect}}
The paper’s objective is to assess the impact of the game designs to the learning outcomes. The three game designs that were observed are game aesthetics, player choice, and player outcomes. The testers in the assessment were told to play a serious game in a span of a week and the result of the assessment is done via pre-test and post-test.

The result showed that aesthetics of the game proved to be a significant factor in the game’s instructional value. This shows that for an educational game to be effective and entertaining, the game’s content needs to be good/sound and the aesthetics should be appealing to the player as well. The player choice and competition did not matter as much.

We will be taking the aesthetics of the application into consideration while developing the game.

\subsection{Design and Implementation of the Game-Design and Learning Program \cite{Implement}}
The paper describes the design and development of the Game Design and Learning (GDL) program, which aims to teach students basic of computer programming ,game design skills, and also teach learners complex problem-solving skills.

The paper mentioned the program's design process, highlighting its potential as a model for similar educational environments and also students who attended the GDL program showed significant improvement in their problem solving skill. However, the paper concluded by encouraging further research on the topic to fully understand the potential of game design for educational purposes

We will take their concepts into consideration when developing our application\newline

\noindent\hspace{1.5em}
We can conclude that game design principles so have significant effects on the learners. Some can provide benefits to the learners while others not so much. After reviewing the papers, we will be taking the elements that can impact the learners positively such as aesthetic and engaging techniques and apply it into the application and possibly consider the others.


\section{Adaptive Learning Algorithms}
\subsection{Algorithm for adaptive learning process and improving learners’ skills in Java programming language \cite{Algo1}}
The paper discussed about an algorithm for an adaptive learning process designed to improve learner's skills in Java programming language.

The author proposed a system consists of learning topics with sections and Java grader tasks to check student's knowledge. The students will start with a basic topic along with grader tasks. The system will tailors the path based on the learner performance by showing relevant sections from the topic or asking learner to repeat the task. 

The system tracks the student progress thought time spent on sections and grader results which will be used to personalize learning path. the system also uses k-Nearest Neighbors algorithm to find similar learners in a test group so that they will be assign to appropriate section.

From this paper, we will be applying the proposed model on our application.
\subsection{A User-Centric Adaptive Learning System for E-Learning 2.0 \cite{Algo2}}
The article talks about a system that personalizes learning path for e-learning users based on their abilities by using methods like sequential pattern mining instead of a traditional adaptive learning systems which provide learning path based on a designers and experts.

The study shows that a user-centric adaptive learning system is as effective as a system designed by an expert. Learners felt more satisfied with it and learned just as well.

We will be taking the system into consideration during the development process.\newline

\noindent\hspace{1.5em}
From the papers above, we analyzed some of the algorithms and we came to the conclusion that some of the algorithms used could be troublesome and complicated to implement into our application while others look plausible. The algorithm mentioned in the first paper of the section seems plausible to be adapted into our application. 