\chapter{LITERATURE REVIEW}

In this chapter, the literature review will be divided into 3 parts: difficulty in learning programming language, game designs and adaptive learning algorithms. We will review the articles and present the advantages, disadvantages, and the summary of the paper. Some points made in the articles will be used to support and improve our project.

\section{Difficulty in Learning Programming Language}
\subsection{Factors Contributing to the Difficulties in Teaching and Learning of Computer Programming: A Literature Review \cite{Factors}}
There are factors contributing to the learning difficulty...
Will write more after finishing reading
\subsection{What about a simple language? Analyzing the difficulties in learning to program \cite{Analyze}}
Simple language gives rise to fewer syntax errors as well as logic errors.
Suggests that not only can a simple language be used when introducing programming as a general skill, but also when providing basic skills to future professionals in the field...

\section{Game Designs}
\subsection{Designing Engaging Games for Education: A Systematic Literature Review on Game Motivators and Design Principles \cite{Design}}
Effective educational interventions require sufficient learner engagement, which can be difficult to achieve if the learner is inadequately motivated...
\subsection{The Effects of Game Design on Learning Outcomes \cite{Effect}}
This article details the administration and results of an experiment conducted to assess the impact of three video game design concepts upon learning outcomes...
\subsection{Design and Implementation of the Game-Design and Learning Program \cite{Implement}}
Design involves solving complex, ill-structured problems. Design tasks are consequently, appropriate contexts for children to exercise higher-order thinking and problem-solving skills...

\section{Adaptive Learning Algorithms}
\subsection{Algorithm for adaptive learning process and improving learners’ skills in Java programming language \cite{Algo1}}
\subsection{A User-Centric Adaptive Learning System for E-Learning 2.0 \cite{Algo2}}
