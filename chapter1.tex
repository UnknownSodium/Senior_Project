\chapter{INTRODUCTION}

\section{Background}
C programming language is known to be some of the most versatile programming language. Not only is it useful for most fields in technology, it is also a base for other languages such as C++ and C\#. The problem of C language is that the language itself is case sensitive and has confusing syntax. This small problem can be a deal breaker for some of the learners if this is their first programming language. We came to the conclusion that by turning the process of learning into something more enjoyable then it might help the learners to ease into the C programming language. We will be achieving this by the process of gamification.

Gamification has been a known concept for a decade now, yet the examples of effective usage are far and few in between. One known example is Duolingo. However, an effective learning application for programming languages has not been developed yet. 

There have been studies on the effectiveness of the gamification of programming languages including a published paper with meta-analysis, though there have been no studies to try and improve the process using the framework suggested by said paper. Our project will be focused on utilizing some parts of the suggested framework.
\newpage

\section{Objective}
\noindent\hspace{1.5em}(Must use action verbs)
\begin{enumerate}
	\item To integrate gamification into programming language learning.
	\item To provide personalized learning experience based on learner's skill and abilities.
\end{enumerate}

\section{Scope}
\begin{enumerate}
    \item The programming language will be C.
    \item The platform that the project will be hosted on is Android.
    \item The number of lessons that will be in the application will be based on the contents of EGCI113 Fundamental Programming.
    % What kind of test/lesson/exam will be in the application?
\end{enumerate}

\section{Expected Results}
\noindent\hspace{1.5em}(Indicate expected outcomes of the project)
\begin{enumerate}
	\item The proposed application will help user focus on learning the language.
	\item The proposed application will aid user's learning journey.
	\item The proposed application will create enjoyable learning experience.
\end{enumerate}

\section{Timeline}
\newcolumntype{L}[1]{>{\raggedright\let\newline\\\arraybackslash\hspace{0pt}}m{#1}}
\newcolumntype{C}[1]{>{\centering\let\newline\\\arraybackslash\hspace{0pt}}m{#1}}
\newcolumntype{R}[1]{>{\raggedleft\let\newline\\\arraybackslash\hspace{0pt}}m{#1}}	

\begin{table}[!ht]
	\footnotesize
	\sloppy
	\centering
	\caption{Project Timeline}
	\label{tab: your-table} %for cross-reference
	\begin{tabular}{|p{2.15cm}|c|c|c|c|c|c|c|c|}
		\hline
		\multicolumn{1}{|c|}{}& \multicolumn{8}{c|}{\textbf{Timeline}} \\ \cline{2-9} 
		\multicolumn{1}{|c|}{}& \multicolumn{8}{c|}{\textbf{2024}}\\ \cline{2-9} 
		\multicolumn{1}{|c|}{\multirow{-3}{2cm}{\textbf{Plan}}} & \textbf{May} & \textbf{Jun} & \textbf{Jul} & \textbf{Aug} & \textbf{Sep} & \textbf{Oct} & \textbf{Nov} & \textbf{Dec} \\ \hline
		1: Study & 
		\cellcolor[HTML]{000000} & & & & & & &\\ \hline
		2: Ch.1 & & 
		\cellcolor[HTML]{000000} & & & & & &\\ \hline
		3: Ch.2 & &  
		\cellcolor[HTML]{000000} & & & & & &\\ \hline
		4: Ch.3 & & &
		\cellcolor[HTML]{000000} &
		\cellcolor[HTML]{000000} &
		\cellcolor[HTML]{3FBB1D} & & &\\ \hline
		5: Ch.4 & & & & &
		\cellcolor[HTML]{000000} &
		\cellcolor[HTML]{000000} & & \\ \hline
		6: Ch.5 & & & & & & &   
		\cellcolor[HTML]{000000} &\\ \hline
		7. Presentation & & & & & & & & 
		\cellcolor[HTML]{000000}\\ \hline
		
	\end{tabular}
\end{table}