\chapter{INTRODUCTION}

\section{Background}
%C programming language is known to be some of the most versatile programming language. Not only is it useful for most fields in technology, it is also a base for other languages such as C++ and C\#. The problem of C language is that the language itself is case sensitive and has confusing syntax. This small problem can be a deal breaker for some of the learners if this is their first programming language. We came to the conclusion that by turning the process of learning into something more enjoyable then it might help the learners to ease into the C programming language.

%Gamification has been a known concept for a decade now, yet the examples of effective usage are far and few in between. One known example is Duolingo. However, an effective learning application for programming languages has not been developed yet. 

%There have been studies on the effectiveness of the gamification of programming languages including a published paper with meta-analysis, though there have been no studies to try and improve the process using the framework suggested by said paper. Our project will be focused on utilizing some parts of the suggested framework.

Programming is a skill that is becoming more popularized and more people are interested in learning how to write code. There are multiple languages for programming, each with different ways to write and its own rules, which can create difficulty for learners. One of the examples is the C programming language with its case sensitivity and syntax. This can cause some people to prefer one language over another. However, all of those languages share a common main problem, high dropout rates among beginners.

According to \citet{LearnDiff}, the main reasons why learners, more specifically undergraduates, have trouble learning their first programming language are due to them not being able to understand how the program works and their low interest in learning caused by lecture-dominant learning in class. \citet{LearnDiff} also found that learners preferred learning via practice rather than lecture. This shows that having an engaging and easy-to-understand learning method can help learners overcome the hurdle of learning their first programming language.
 
Many researchers have tried to find a solution to this problem, for example, block coding. \citet{BlockCode} did a case study of this method of coding with students in Columbia and found that while it does help engage the students in learning their programming language, students still have problems with translating it into line coding. This suggests that having variation in exercise could help learners avoid this problem. We have come to the conclusion that by turning the lessons into a game format, also known as gamification, with different types of exercises could possibly help learners understand and write code better. There are researchers who have provided the framework for effective gamification but only in theory. We will be applying those frameworks to observe whether gamification can aid learners in learning programming languages.

\section{Objective}
\noindent\hspace{1.5em}(Must use action verbs)
\begin{enumerate}
	\item To integrate gamification into programming language learning.
	\item To provide personalized learning experience based on learner's skill and abilities.
\end{enumerate}

\section{Scope}
\begin{enumerate}
    \item The programming language will be C.
    \item The platform that the project will be hosted on is Android.
    \item The number of lessons that will be in the application will be based on the contents of EGCI113 Fundamental Programming.
    % What kind of test/lesson/exam will be in the application?
\end{enumerate}

\section{Expected Results}
\noindent\hspace{1.5em}(Indicate expected outcomes of the project)
\begin{enumerate}
	\item The proposed application will help user focus on learning the language.
	\item The proposed application will aid user's learning journey.
	\item The proposed application will create enjoyable learning experience.
\end{enumerate}

\section{Timeline}
\newcolumntype{L}[1]{>{\raggedright\let\newline\\\arraybackslash\hspace{0pt}}m{#1}}
\newcolumntype{C}[1]{>{\centering\let\newline\\\arraybackslash\hspace{0pt}}m{#1}}
\newcolumntype{R}[1]{>{\raggedleft\let\newline\\\arraybackslash\hspace{0pt}}m{#1}}	

\begin{table}[!ht]
	\footnotesize
	\sloppy
	\centering
	\caption{Project Timeline}
	\label{tab: your-table} %for cross-reference
	\begin{tabular}{|p{2.15cm}|c|c|c|c|c|c|c|c|}
		\hline
		\multicolumn{1}{|c|}{}& \multicolumn{8}{c|}{\textbf{Timeline}} \\ \cline{2-9} 
		\multicolumn{1}{|c|}{}& \multicolumn{8}{c|}{\textbf{2024}}\\ \cline{2-9} 
		\multicolumn{1}{|c|}{\multirow{-3}{2cm}{\textbf{Plan}}} & \textbf{May} & \textbf{Jun} & \textbf{Jul} & \textbf{Aug} & \textbf{Sep} & \textbf{Oct} & \textbf{Nov} & \textbf{Dec} \\ \hline
		1: Study & 
		\cellcolor[HTML]{000000} & & & & & & &\\ \hline
		2: Ch.1 & & 
		\cellcolor[HTML]{000000} & & & & & &\\ \hline
		3: Ch.2 & &  
		\cellcolor[HTML]{000000} & & & & & &\\ \hline
		4: Ch.3 & & &
		\cellcolor[HTML]{000000} &
		\cellcolor[HTML]{000000} &
		\cellcolor[HTML]{3FBB1D} & & &\\ \hline
		5: Ch.4 & & & & &
		\cellcolor[HTML]{000000} &
		\cellcolor[HTML]{000000} & & \\ \hline
		6: Ch.5 & & & & & & &   
		\cellcolor[HTML]{000000} &\\ \hline
		7. Presentation & & & & & & & & 
		\cellcolor[HTML]{000000}\\ \hline
		
	\end{tabular}
\end{table}
